%% Generated by Sphinx.
\def\sphinxdocclass{report}
\documentclass[letterpaper,10pt,english]{sphinxmanual}
\ifdefined\pdfpxdimen
   \let\sphinxpxdimen\pdfpxdimen\else\newdimen\sphinxpxdimen
\fi \sphinxpxdimen=.75bp\relax

\usepackage[utf8]{inputenc}
\ifdefined\DeclareUnicodeCharacter
 \ifdefined\DeclareUnicodeCharacterAsOptional
  \DeclareUnicodeCharacter{"00A0}{\nobreakspace}
  \DeclareUnicodeCharacter{"2500}{\sphinxunichar{2500}}
  \DeclareUnicodeCharacter{"2502}{\sphinxunichar{2502}}
  \DeclareUnicodeCharacter{"2514}{\sphinxunichar{2514}}
  \DeclareUnicodeCharacter{"251C}{\sphinxunichar{251C}}
  \DeclareUnicodeCharacter{"2572}{\textbackslash}
 \else
  \DeclareUnicodeCharacter{00A0}{\nobreakspace}
  \DeclareUnicodeCharacter{2500}{\sphinxunichar{2500}}
  \DeclareUnicodeCharacter{2502}{\sphinxunichar{2502}}
  \DeclareUnicodeCharacter{2514}{\sphinxunichar{2514}}
  \DeclareUnicodeCharacter{251C}{\sphinxunichar{251C}}
  \DeclareUnicodeCharacter{2572}{\textbackslash}
 \fi
\fi
\usepackage{cmap}
\usepackage[T1]{fontenc}
\usepackage{amsmath,amssymb,amstext}
\usepackage{babel}
\usepackage{times}
\usepackage[Bjarne]{fncychap}
\usepackage[dontkeepoldnames]{sphinx}

\usepackage{geometry}

% Include hyperref last.
\usepackage{hyperref}
% Fix anchor placement for figures with captions.
\usepackage{hypcap}% it must be loaded after hyperref.
% Set up styles of URL: it should be placed after hyperref.
\urlstyle{same}

\addto\captionsenglish{\renewcommand{\figurename}{Fig.}}
\addto\captionsenglish{\renewcommand{\tablename}{Table}}
\addto\captionsenglish{\renewcommand{\literalblockname}{Listing}}

\addto\captionsenglish{\renewcommand{\literalblockcontinuedname}{continued from previous page}}
\addto\captionsenglish{\renewcommand{\literalblockcontinuesname}{continues on next page}}

\addto\extrasenglish{\def\pageautorefname{page}}

\setcounter{tocdepth}{0}



\title{pycropml Documentation}
\date{May 02, 2018}
\release{0.0.2}
\author{Cyrille Ahmed Midingoyi}
\newcommand{\sphinxlogo}{\vbox{}}
\renewcommand{\releasename}{Release}
\makeindex

\begin{document}

\maketitle
\sphinxtableofcontents
\phantomsection\label{\detokenize{index::doc}}



\chapter{\sphinxstylestrong{Contents}}
\label{\detokenize{index:welcome-to-pycropml-documentation}}\label{\detokenize{index:contents}}
\begin{sphinxShadowBox}
\sphinxstylesidebartitle{Summary}
\begin{quote}\begin{description}
\item[{Version}] \leavevmode
0.0.2

\item[{Release}] \leavevmode
0.0.2

\item[{Date}] \leavevmode
May 02, 2018

\item[{Author}] \leavevmode
See {\color{red}\bfseries{}{}`authors{}`\_} section

\item[{ChangeLog}] \leavevmode
See {\color{red}\bfseries{}{}`changelog{}`\_} section

\end{description}\end{quote}
\end{sphinxShadowBox}


\section{What is PyCropML?}
\label{\detokenize{user/overview:what-is-pycropml}}\label{\detokenize{user/overview::doc}}
\begin{DUlineblock}{0em}
\item[] \sphinxstylestrong{PyCropML} is a free, open-source library for defining
and exchanging CropML models.It is used to generate components
of modeling and simulation platforms from the CropML specification and
allow component exchange between different platform.
\item[] It allows to parse the models described in CropML format
and automatically generate the equivalent executable Python, java, C\#, C++ components
and packages usable from existing crop simulation platform.
\end{DUlineblock}


\subsection{What is CropML ?}
\label{\detokenize{user/overview:what-is-cropml}}
\begin{DUlineblock}{0em}
\item[] \sphinxstylestrong{CropML} is a language based on XML format that allows
to represent different biological processes involved
in the crop models.
\item[] CropML project is intented to provide
a common framework for defining and exchanging descriptions
of crop growth models between crop simulation frameworks.
\end{DUlineblock}


\subsection{Objectives}
\label{\detokenize{user/overview:objectives}}
Our main objectives are:
\begin{itemize}
\item {} 
define a \sphinxstylestrong{declarative language} to describe either an atomic model or a composition of models

\item {} 
add semantic dimension to CropML language by annotation of the models to allow the composition of components of different platforms by using the standards of the semantic web

\item {} 
develop a library to allow the transformation and the exchange of CropML model between different Crop modelling and simulation platform

\item {} 
provide a \sphinxstylestrong{web repository} enabling registration, search and discovery of CropML Models

\item {} 
facilitate Agricultural Model Exchange Initiative

\end{itemize}


\subsection{Context}
\label{\detokenize{user/overview:context}}
\begin{DUlineblock}{0em}
\item[] Nowadays, we observe the emergence of plant growth models which are built
in differents  platforms. Although standard platform development initiatives
are emerged, there is a lack of  transparency, reusability, and exchange
code between platforms due to the high diversity of modeling languages
leading to a lack of benchmarking between the different platforms.
\item[] This project aims to gather developers and plant growth modellers
to define a standard framework based on the development of declarative language and libraries to improve exchange model components between platforms.
\end{DUlineblock}


\subsection{Motivation}
\label{\detokenize{user/overview:motivation}}\begin{itemize}
\item {} 
Facilitate model intercomparison (at the process level) and model improvement through the exchange of model components (algorithms) and code reuse between platforms/models.

\item {} 
Bridge the gap between ecophysiologists who develop models at the process level with crop modelers and model users and facilitate the integration in crop models of new knowledge in plant science (i.e. we are seeking the exchange of knowledge rather than black box models).

\item {} 
Increase capabilities and responsiveness to stakeholder’ needs.

\item {} 
Propose a solution to the AgMIP community for NexGen crop modeling tools.

\end{itemize}


\subsection{Vision}
\label{\detokenize{user/overview:vision}}\begin{itemize}
\item {} 
Use modular modelling to share knowledge and rapidly develop operational tools.

\item {} 
Reuse model parts to leverage the expertise of third parties;

\item {} 
Renovate legacy code.

\item {} 
Realize the benefit of sharing and complementing different expertise.

\end{itemize}


\section{\sphinxstylestrong{CropML Description}}
\label{\detokenize{user/description:cropml-description}}\label{\detokenize{user/description::doc}}
\begin{DUlineblock}{0em}
\item[] In CropML, a model is either a model unit or a composition of models. A ModelUnit  represents the atomic unit of a crop model define by the
modelers. A model composition  is a model resulting from the composition of two or more atomic model.
\item[] These models have a specific formal definition in CropML.
\end{DUlineblock}


\subsection{Formal definition of a Model Unit in CropML}
\label{\detokenize{user/description:formal-definition-of-a-model-unit-in-cropml}}
\begin{DUlineblock}{0em}
\item[] The structure of a Model Unit in CropML MUST be conform to a specific Document Type Definition
named \sphinxhref{https://github.com/AgriculturalModelExchangeInitiative/PyCropML/blob/version2/test/data/ModelUnit.dtd}{ModelUnit.dtd} .
\item[] So a Model Unit CropML document is a XML document well-formed and also obeys the rules given in the ModelUnit schema.
\item[] This structure MAY be described by the below tree:
\end{DUlineblock}

\noindent\sphinxincludegraphics{{modelunit}.png}

\begin{DUlineblock}{0em}
\item[] In the next, we define the major elements of a CropML model unit.
\end{DUlineblock}


\subsubsection{ModelUnit element}
\label{\detokenize{user/description:modelunit-element}}
\begin{DUlineblock}{0em}
\item[] An atomic model in CropML is declared with \sphinxhref{https://github.com/AgriculturalModelExchangeInitiative/PyCropML/blob/master/src/pycropml/modelunit.py}{\textless{}\textless{}ModelUnit\textgreater{}\textgreater{}} element,
the usual root of CropML ModelUnit document.
\end{DUlineblock}

\fvset{hllines={, ,}}%
\begin{sphinxVerbatim}[commandchars=\\\{\}]
\PYG{c+cp}{\PYGZlt{}?xml version=\PYGZdq{}1.0\PYGZdq{} encoding=\PYGZdq{}utf\PYGZhy{}8\PYGZdq{}?\PYGZgt{}}
\PYG{c+cp}{\PYGZlt{}!DOCTYPE ModelUnit PUBLIC \PYGZdq{}\PYGZhy{}//SIMPLACE/DTD SOL 1.0//EN\PYGZdq{} \PYGZdq{}https://raw.githubusercontent.com/AgriculturalModelExchangeInitiative/xml\PYGZus{}representation/master/ModelUnit.dtd\PYGZdq{}\PYGZgt{}}
\PYG{n+nt}{\PYGZlt{}ModelUnit} \PYG{n+na}{modelid=}\PYG{l+s}{\PYGZdq{} \PYGZdq{}} \PYG{n+na}{timestep=}\PYG{l+s}{\PYGZdq{} \PYGZdq{}} \PYG{n+na}{name=}\PYG{l+s}{\PYGZdq{} \PYGZdq{}} \PYG{n+na}{version=}\PYG{l+s}{\PYGZdq{}\PYGZdq{}}\PYG{n+nt}{\PYGZgt{}}
        ....
\PYG{n+nt}{\PYGZlt{}/ModelUnit\PYGZgt{}}
\end{sphinxVerbatim}

\begin{DUlineblock}{0em}
\item[] This element MUST contain a Description, an Algorithm, Parametersets and Testsets elements and
MAY optionally have Inputs and Outputs elements. The restriction of the length of different lists is not imposed.
\item[] ModelUnit element MUST have an modelid and name attributes which are used to reference an atomic model. It MUST also contain a timestep attribute to define the temporality of the model and
a version attribute for each version of the model.
\end{DUlineblock}


\subsubsection{Description element}
\label{\detokenize{user/description:description-element}}
This element gives the general information on the model and is composed by a set of character elements. It MUST contain
Title, Authors, Institution and abstract elements and MAY optionally contain URI and Reference elements.

\fvset{hllines={, ,}}%
\begin{sphinxVerbatim}[commandchars=\\\{\}]
\PYG{n+nt}{\PYGZlt{}ModelUnit} \PYG{n+na}{modelid=}\PYG{l+s}{\PYGZdq{} \PYGZdq{}} \PYG{n+na}{timestep=}\PYG{l+s}{\PYGZdq{} \PYGZdq{}} \PYG{n+na}{name=}\PYG{l+s}{\PYGZdq{} \PYGZdq{}} \PYG{n+na}{version =}\PYG{l+s}{\PYGZdq{} \PYGZdq{}}\PYG{n+nt}{\PYGZgt{}}
        \PYG{n+nt}{\PYGZlt{}Description}\PYG{n+nt}{\PYGZgt{}}
                \PYG{n+nt}{\PYGZlt{}Title}\PYG{n+nt}{\PYGZgt{}}title\PYG{n+nt}{\PYGZlt{}/Title\PYGZgt{}}
                \PYG{n+nt}{\PYGZlt{}Authors}\PYG{n+nt}{\PYGZgt{}}authors\PYG{n+nt}{\PYGZlt{}/Authors\PYGZgt{}}
                \PYG{n+nt}{\PYGZlt{}Institution}\PYG{n+nt}{\PYGZgt{}}institution\PYG{n+nt}{\PYGZlt{}/Institution\PYGZgt{}}
                \PYG{n+nt}{\PYGZlt{}URI}\PYG{n+nt}{\PYGZgt{}}uri\PYG{n+nt}{\PYGZlt{}/URI\PYGZgt{}}
                \PYG{n+nt}{\PYGZlt{}Abstract}\PYG{n+nt}{\PYGZgt{}}\PYG{c+cp}{\PYGZlt{}![CDATA[abstract]]\PYGZgt{}}\PYG{n+nt}{\PYGZlt{}/Abstract\PYGZgt{}}
        \PYG{n+nt}{\PYGZlt{}/Description\PYGZgt{}}
        ...
\PYG{n+nt}{\PYGZlt{}/ModelUnit\PYGZgt{}}
\end{sphinxVerbatim}


\subsubsection{Inputs elements}
\label{\detokenize{user/description:inputs-elements}}
The inputs of Model are listed inside an XML element called Inputs within a \sphinxhref{https://github.com/AgriculturalModelExchangeInitiative/PyCropML/blob/version2/src/pycropml/inout.py}{dictionary structure}
composed by their attributes which declarations are optional(default, max, min, parametercategory, variablecategory and uri) or required(name, datatype, description, inputtype,
unit ) and their corresponding value. \sphinxstyleemphasis{Inputs} element MUST contain one or more \sphinxstyleemphasis{Input} elements.

\fvset{hllines={, ,}}%
\begin{sphinxVerbatim}[commandchars=\\\{\}]
\PYG{n+nt}{\PYGZlt{}ModelUnit} \PYG{n+na}{modelid=}\PYG{l+s}{\PYGZdq{} \PYGZdq{}} \PYG{n+na}{timestep=}\PYG{l+s}{\PYGZdq{} \PYGZdq{}} \PYG{n+na}{name=}\PYG{l+s}{\PYGZdq{} \PYGZdq{}} \PYG{n+na}{version =}\PYG{l+s}{\PYGZdq{} \PYGZdq{}}\PYG{n+nt}{\PYGZgt{}}
   ...
   \PYG{n+nt}{\PYGZlt{}Inputs}\PYG{n+nt}{\PYGZgt{}}
      \PYG{n+nt}{\PYGZlt{}Input} \PYG{n+na}{name=}\PYG{l+s}{\PYGZdq{} \PYGZdq{}} \PYG{n+na}{description=}\PYG{l+s}{\PYGZdq{} \PYGZdq{}} \PYG{n+na}{parametercategory=}\PYG{l+s}{\PYGZdq{} \PYGZdq{}} \PYG{n+na}{datatype=}\PYG{l+s}{\PYGZdq{} \PYGZdq{}} \PYG{n+na}{min=}\PYG{l+s}{\PYGZdq{} \PYGZdq{}} \PYG{n+na}{max=}\PYG{l+s}{\PYGZdq{} \PYGZdq{}} \PYG{n+na}{default=}\PYG{l+s}{\PYGZdq{} \PYGZdq{}} \PYG{n+na}{unit=}\PYG{l+s}{\PYGZdq{} \PYGZdq{}} \PYG{n+na}{uri=}\PYG{l+s}{\PYGZdq{}\PYGZdq{}} \PYG{n+na}{inputtype=}\PYG{l+s}{\PYGZdq{} \PYGZdq{}}\PYG{n+nt}{/\PYGZgt{}}
      \PYG{n+nt}{\PYGZlt{}Input} \PYG{n+na}{name=}\PYG{l+s}{\PYGZdq{} \PYGZdq{}} \PYG{n+na}{description=}\PYG{l+s}{\PYGZdq{} \PYGZdq{}} \PYG{n+na}{parametercategory=}\PYG{l+s}{\PYGZdq{} \PYGZdq{}} \PYG{n+na}{datatype=}\PYG{l+s}{\PYGZdq{} \PYGZdq{}} \PYG{n+na}{min=}\PYG{l+s}{\PYGZdq{} \PYGZdq{}} \PYG{n+na}{max=}\PYG{l+s}{\PYGZdq{} \PYGZdq{}} \PYG{n+na}{default=}\PYG{l+s}{\PYGZdq{} \PYGZdq{}} \PYG{n+na}{unit=}\PYG{l+s}{\PYGZdq{} \PYGZdq{}} \PYG{n+na}{uri=}\PYG{l+s}{\PYGZdq{} \PYGZdq{}} \PYG{n+na}{inputtype=}\PYG{l+s}{\PYGZdq{} \PYGZdq{}}\PYG{n+nt}{/\PYGZgt{}}
      ...
   \PYG{n+nt}{\PYGZlt{}/Inputs\PYGZgt{}}
   ...
\PYG{n+nt}{\PYGZlt{}/ModelUnit\PYGZgt{}}
\end{sphinxVerbatim}
\begin{itemize}
\item {} 
The required \sphinxstyleemphasis{datatype} attribute is the type of input value specified in \sphinxstyleemphasis{default} (the default value in the input), \sphinxstyleemphasis{min} (the minimum value in the input) and \sphinxstyleemphasis{max} (the maximum value in the input). It MAY be one type of the set of types used in the existing crop modeling platform.

\item {} 
The \sphinxstyleemphasis{inputtype} attribute makes it possible to distinguish the variables and the parameters of the model. So it MUST take one of two possible values: \sphinxstyleemphasis{parameter} and \sphinxstyleemphasis{variable}.

\item {} 
The \sphinxstyleemphasis{parametercategory} attribute defines the category of parameter which is specified by one of the following values: \sphinxstyleemphasis{constant}, \sphinxstyleemphasis{species}, \sphinxstyleemphasis{soil} and \sphinxstyleemphasis{genotypic}.

\item {} 
The \sphinxstyleemphasis{variablecategory} defines the category of variable depending on whether it is a \sphinxstyleemphasis{state}, a \sphinxstyleemphasis{rate} or an “auxiliary” variable.  State variable characterize the behavior of the model and rate variable characterizes the changes in state variables.

\end{itemize}


\subsubsection{Outputs element}
\label{\detokenize{user/description:outputs-element}}
The outputs of Model are listed inside an XML element called Outputs within a \sphinxhref{https://github.com/AgriculturalModelExchangeInitiative/PyCropML/blob/version2/src/pycropml/inout.py}{dictionary structure}
composed by their attributes which declarations are:
\begin{itemize}
\item {} 
optional(variabletype and URI)

\item {} 
required(name, datatype, description, unit, max and min )

\item {} 
and their corresponding value

\end{itemize}

\sphinxstyleemphasis{Outputs} MUST contain zero or more output elements.

\fvset{hllines={, ,}}%
\begin{sphinxVerbatim}[commandchars=\\\{\}]
\PYG{n+nt}{\PYGZlt{}ModelUnit} \PYG{n+na}{modelid=}\PYG{l+s}{\PYGZdq{} \PYGZdq{}} \PYG{n+na}{timestep=}\PYG{l+s}{\PYGZdq{} \PYGZdq{}} \PYG{n+na}{name=}\PYG{l+s}{\PYGZdq{} \PYGZdq{}} \PYG{n+na}{version =}\PYG{l+s}{\PYGZdq{} \PYGZdq{}}\PYG{n+nt}{\PYGZgt{}}
   ...
   \PYG{n+nt}{\PYGZlt{}Outputs}\PYG{n+nt}{\PYGZgt{}}
      \PYG{n+nt}{\PYGZlt{}Output} \PYG{n+na}{name=}\PYG{l+s}{\PYGZdq{}\PYGZdq{}} \PYG{n+na}{description=}\PYG{l+s}{\PYGZdq{}\PYGZdq{}} \PYG{n+na}{datatype=}\PYG{l+s}{\PYGZdq{}float\PYGZdq{}} \PYG{n+na}{min=}\PYG{l+s}{\PYGZdq{}\PYGZdq{}} \PYG{n+na}{max=}\PYG{l+s}{\PYGZdq{}\PYGZdq{}}  \PYG{n+na}{unit=}\PYG{l+s}{\PYGZdq{}\PYGZdq{}} \PYG{n+na}{uri=}\PYG{l+s}{\PYGZdq{}\PYGZdq{}}\PYG{n+nt}{/\PYGZgt{}}
      \PYG{n+nt}{\PYGZlt{}Output} \PYG{n+na}{name=}\PYG{l+s}{\PYGZdq{}\PYGZdq{}} \PYG{n+na}{description=}\PYG{l+s}{\PYGZdq{}\PYGZdq{}} \PYG{n+na}{datatype=}\PYG{l+s}{\PYGZdq{}float\PYGZdq{}} \PYG{n+na}{min=}\PYG{l+s}{\PYGZdq{}\PYGZdq{}} \PYG{n+na}{max=}\PYG{l+s}{\PYGZdq{}\PYGZdq{}}  \PYG{n+na}{unit=}\PYG{l+s}{\PYGZdq{}\PYGZdq{}} \PYG{n+na}{uri=}\PYG{l+s}{\PYGZdq{}\PYGZdq{}}\PYG{n+nt}{/\PYGZgt{}}
      ...
   \PYG{n+nt}{\PYGZlt{}/Outputs\PYGZgt{}}
   ...
\PYG{n+nt}{\PYGZlt{}/ModelUnit\PYGZgt{}}
\end{sphinxVerbatim}

\begin{DUlineblock}{0em}
\item[] The definition of different attributes is same as Input’s attributes.
\end{DUlineblock}


\subsubsection{Algorithm element}
\label{\detokenize{user/description:algorithm-element}}
\begin{DUlineblock}{0em}
\item[] The \sphinxstyleemphasis{Algorithm} element defines the building block of CropML model unit and shows the computational method to determine
the outputs from the inputs.
\item[] It consists of a set of mathematical equations (relation between inputs), loops and conditional instructions
which are well structured in a specific \sphinxstyleemphasis{language}, the algorithm’s attribute.
\end{DUlineblock}

\fvset{hllines={, ,}}%
\begin{sphinxVerbatim}[commandchars=\\\{\}]
\PYG{n+nt}{\PYGZlt{}ModelUnit} \PYG{n+na}{modelid=}\PYG{l+s}{\PYGZdq{} \PYGZdq{}} \PYG{n+na}{timestep=}\PYG{l+s}{\PYGZdq{} \PYGZdq{}} \PYG{n+na}{name=}\PYG{l+s}{\PYGZdq{} \PYGZdq{}} \PYG{n+na}{version =}\PYG{l+s}{\PYGZdq{} \PYGZdq{}}\PYG{n+nt}{\PYGZgt{}}
   ...
   \PYG{n+nt}{\PYGZlt{}Algorithm} \PYG{n+na}{language =}\PYG{l+s}{\PYGZdq{}\PYGZdq{}}\PYG{n+nt}{\PYGZgt{}}\PYG{c+cp}{\PYGZlt{}![CDATA[}
\PYG{c+cp}{      ...}
\PYG{c+cp}{      ]]\PYGZgt{}}
   \PYG{n+nt}{\PYGZlt{}/Algorithm\PYGZgt{}}
   ...
\PYG{n+nt}{\PYGZlt{}/ModelUnit\PYGZgt{}}
\end{sphinxVerbatim}


\subsubsection{Parametersets element}
\label{\detokenize{user/description:parametersets-element}}
\begin{DUlineblock}{0em}
\item[] \sphinxstyleemphasis{Parametersets} element contains one or more \sphinxstyleemphasis{Parameterset} elements that define the different ways of setting the model.
Each \sphinxstyleemphasis{Parameterset} element MUST have \sphinxstyleemphasis{name} and \sphinxstyleemphasis{description} attributes that respectively represents the name and the description of each setting.
\end{DUlineblock}

\begin{DUlineblock}{0em}
\item[] The different parameterset MUST contain a list of Param elements that show in attribute the name of the parameter (an input
which inputtype equals \sphinxstyleemphasis{parameter}) and the fixed value of this one.
\end{DUlineblock}

\fvset{hllines={, ,}}%
\begin{sphinxVerbatim}[commandchars=\\\{\}]
\PYG{n+nt}{\PYGZlt{}ModelUnit} \PYG{n+na}{modelid=}\PYG{l+s}{\PYGZdq{} \PYGZdq{}} \PYG{n+na}{timestep=}\PYG{l+s}{\PYGZdq{} \PYGZdq{}} \PYG{n+na}{name=}\PYG{l+s}{\PYGZdq{} \PYGZdq{}} \PYG{n+na}{version =}\PYG{l+s}{\PYGZdq{} \PYGZdq{}}\PYG{n+nt}{\PYGZgt{}}
    ...
    \PYG{n+nt}{\PYGZlt{}Parametersets}\PYG{n+nt}{\PYGZgt{}}
       \PYG{n+nt}{\PYGZlt{}Parameterset} \PYG{n+na}{name=}\PYG{l+s}{\PYGZdq{}\PYGZdq{}} \PYG{n+na}{description=}\PYG{l+s}{\PYGZdq{}\PYGZdq{}} \PYG{n+na}{uri =} \PYG{l+s}{\PYGZdq{}\PYGZdq{}}\PYG{n+nt}{/\PYGZgt{}}
       \PYG{n+nt}{\PYGZlt{}Parameterset} \PYG{n+na}{name=}\PYG{l+s}{\PYGZdq{}\PYGZdq{}} \PYG{n+na}{description=}\PYG{l+s}{\PYGZdq{}\PYGZdq{}} \PYG{n+nt}{\PYGZgt{}}
          \PYG{n+nt}{\PYGZlt{}Param} \PYG{n+na}{name=}\PYG{l+s}{\PYGZdq{}\PYGZdq{}}\PYG{n+nt}{\PYGZgt{}}value\PYG{n+nt}{\PYGZlt{}/Param\PYGZgt{}}
          \PYG{n+nt}{\PYGZlt{}Param} \PYG{n+na}{name=}\PYG{l+s}{\PYGZdq{}\PYGZdq{}}\PYG{n+nt}{\PYGZgt{}}value\PYG{n+nt}{\PYGZlt{}/Param\PYGZgt{}}
          ...
       \PYG{n+nt}{\PYGZlt{}/Parameterset\PYGZgt{}}
       ...
    ...
 \PYG{n+nt}{\PYGZlt{}/ModelUnit\PYGZgt{}}
\end{sphinxVerbatim}


\subsubsection{Testsets element}
\label{\detokenize{user/description:testsets-element}}
\begin{DUlineblock}{0em}
\item[] \sphinxstyleemphasis{Testsets} element contains one or more \sphinxstyleemphasis{Testset} elements that define the different run for evaluating the outputs of the model.
\end{DUlineblock}

\begin{DUlineblock}{0em}
\item[] Each \sphinxstyleemphasis{Testset} element MUST have \sphinxstyleemphasis{name}, \sphinxstyleemphasis{description} and \sphinxstyleemphasis{parameterset} attributes that respectively represents the name,
the description of each run and the name of the parameterset related to the Testset. This one allow to retrieve the name and the value of different
parameters includes in this parameterset.
\end{DUlineblock}

\begin{DUlineblock}{0em}
\item[] The different Testset MUST contain a list of InputValue and OutputValue elements corresponding respectively to the values
of inputs used in the run and the values of Outputs that will be asserted.
\end{DUlineblock}

\fvset{hllines={, ,}}%
\begin{sphinxVerbatim}[commandchars=\\\{\}]
\PYG{n+nt}{\PYGZlt{}ModelUnit} \PYG{n+na}{modelid=}\PYG{l+s}{\PYGZdq{} \PYGZdq{}} \PYG{n+na}{timestep=}\PYG{l+s}{\PYGZdq{} \PYGZdq{}} \PYG{n+na}{name=}\PYG{l+s}{\PYGZdq{} \PYGZdq{}} \PYG{n+na}{version =}\PYG{l+s}{\PYGZdq{} \PYGZdq{}}\PYG{n+nt}{\PYGZgt{}}
   ...
   \PYG{n+nt}{\PYGZlt{}Testsets}\PYG{n+nt}{\PYGZgt{}}
      \PYG{n+nt}{\PYGZlt{}Testset} \PYG{n+na}{name=}\PYG{l+s}{\PYGZdq{}\PYGZdq{}} \PYG{n+na}{parameterset =} \PYG{l+s}{\PYGZdq{}\PYGZdq{}} \PYG{n+na}{description=}\PYG{l+s}{\PYGZdq{}\PYGZdq{}} \PYG{n+na}{uri =} \PYG{l+s}{\PYGZdq{}\PYGZdq{}}\PYG{n+nt}{/\PYGZgt{}}
      \PYG{n+nt}{\PYGZlt{}Testset} \PYG{n+na}{name=}\PYG{l+s}{\PYGZdq{}\PYGZdq{}} \PYG{n+na}{parameterset =} \PYG{l+s}{\PYGZdq{}\PYGZdq{}} \PYG{n+na}{description=}\PYG{l+s}{\PYGZdq{}\PYGZdq{}} \PYG{n+nt}{\PYGZgt{}}
         \PYG{n+nt}{\PYGZlt{}Test} \PYG{n+na}{name=}\PYG{l+s}{\PYGZdq{}\PYGZdq{}}\PYG{n+nt}{\PYGZgt{}}
            \PYG{n+nt}{\PYGZlt{}InputValue} \PYG{n+na}{name=}\PYG{l+s}{\PYGZdq{}\PYGZdq{}}\PYG{n+nt}{\PYGZgt{}}value\PYG{n+nt}{\PYGZlt{}/InputValue\PYGZgt{}}
            ...
            \PYG{n+nt}{\PYGZlt{}OutputValue} \PYG{n+na}{name=}\PYG{l+s}{\PYGZdq{}\PYGZdq{}} \PYG{n+na}{precision =}\PYG{l+s}{\PYGZdq{}\PYGZdq{}}\PYG{n+nt}{\PYGZgt{}}value\PYG{n+nt}{\PYGZlt{}/OutputValue\PYGZgt{}}
            ...
         \PYG{n+nt}{\PYGZlt{}/Test\PYGZgt{}}
         ...
      \PYG{n+nt}{\PYGZlt{}/Testset\PYGZgt{}}
      ...
  \PYG{n+nt}{\PYGZlt{}/Testsets\PYGZgt{}}
  ...
\PYG{n+nt}{\PYGZlt{}/ModelUnit\PYGZgt{}}
\end{sphinxVerbatim}


\subsection{Formal definition of a Composite Model in CropML}
\label{\detokenize{user/description:formal-definition-of-a-composite-model-in-cropml}}
\begin{DUlineblock}{0em}
\item[] A Composite Model CropML is an assembly of processes which are described by a set of model units or a composite model.
Given a composite model is a model, this one has also inputs, outputs and internal state which describe the orchestration of different
independent models composed.
\end{DUlineblock}

\begin{DUlineblock}{0em}
\item[] The structure of a Composite Model in CropML MUST be conform to a specific Document Type Definition
named \sphinxhref{https://github.com/AgriculturalModelExchangeInitiative/PyCropML/blob/version2/test/data/ModelComposition.dtd}{ModelComposition.dtd} .
\end{DUlineblock}

\begin{DUlineblock}{0em}
\item[] The composition is represented as a directed port graph of models:
\end{DUlineblock}
\begin{quote}

\begin{DUlineblock}{0em}
\item[] Vertices are the different models that form the composition.
\item[] Ports are the inputs and outputs of each model.
\item[] Edges are directed and connect one output port to an input port of another model.
\end{DUlineblock}
\end{quote}

\begin{DUlineblock}{0em}
\item[] It contains in addition to all Elements of a model unit a Composition Element for the composition of models.
\item[] This structure MAY be described by the below tree:
\end{DUlineblock}

\noindent\sphinxincludegraphics{{modelcomposition}.png}

\begin{DUlineblock}{0em}
\item[] In the next, we define the major elements of a CropML model unit.
\end{DUlineblock}


\subsubsection{Inputs element}
\label{\detokenize{user/description:inputs-element}}
It MUST contain one or more \sphinxstyleemphasis{input} element which provide a set of independent models entries.
If two or more input variables of independent models are the same (same unit, interval, description)
a link should be made to one input variable of the composite model.


\subsubsection{Outputs element}
\label{\detokenize{user/description:id2}}
It MUST contain one or more \sphinxstyleemphasis{output} element which provide a set of independent models outputs or a result of a combination of models .


\subsubsection{Composition element}
\label{\detokenize{user/description:composition-element}}
It’s a list of \sphinxstyleemphasis{models} elements which contains a list of \sphinxstyleemphasis{links} elements.
Link provides the mechanism for mapping inputs declared within one modelUnit to output in another modelUnit,
allowing information to be exchanged between the various atomic models in the composite model.


\section{PyCropML User Guide}
\label{\detokenize{user/index:pycropml-user-guide}}\label{\detokenize{user/index:pycropml-user}}\label{\detokenize{user/index::doc}}\begin{quote}\begin{description}
\item[{Version}] \leavevmode
0.0.2

\item[{Release}] \leavevmode
0.0.2

\item[{Date}] \leavevmode
Apr 30, 2018

\end{description}\end{quote}

This reference manual details functions, modules, and objects included in
OpenAlea.Core, describing what they are and what they do. For a complete
reference guide, see \DUrole{xref,std,std-ref}{core\_reference}.

\begin{sphinxadmonition}{warning}{Warning:}
This “Reference Guide” is still very much in progress.
Many aspects of OpenAlea.Core are not covered.
\end{sphinxadmonition}


\subsection{Manual}
\label{\detokenize{user/manual:manual}}\label{\detokenize{user/manual::doc}}
\begin{sphinxadmonition}{note}{Note:}
The following examples assume you have installed the packages and setup your python path correctly.
\end{sphinxadmonition}


\subsubsection{Installation}
\label{\detokenize{user/manual:installation}}
\fvset{hllines={, ,}}%
\begin{sphinxVerbatim}[commandchars=\\\{\}]
\PYG{n}{conda} \PYG{n}{install} \PYG{o}{\PYGZhy{}}\PYG{n}{c} \PYG{n}{openalea} \PYG{n}{pycropml}
\end{sphinxVerbatim}

or

\fvset{hllines={, ,}}%
\begin{sphinxVerbatim}[commandchars=\\\{\}]
\PYG{n}{python} \PYG{n}{setup}\PYG{o}{.}\PYG{n}{py} \PYG{n}{install}
\end{sphinxVerbatim}


\subsubsection{Overview of the different classes}
\label{\detokenize{user/manual:overview-of-the-different-classes}}

\section{src}
\label{\detokenize{_dvlpt/modules:src}}\label{\detokenize{_dvlpt/modules::doc}}

\subsection{pycropml package}
\label{\detokenize{_dvlpt/pycropml::doc}}\label{\detokenize{_dvlpt/pycropml:pycropml-package}}

\subsubsection{Submodules}
\label{\detokenize{_dvlpt/pycropml:submodules}}

\subsubsection{pycropml.algorithm module}
\label{\detokenize{_dvlpt/pycropml:pycropml-algorithm-module}}\label{\detokenize{_dvlpt/pycropml:module-pycropml.algorithm}}\index{pycropml.algorithm (module)}\index{Algorithm (class in pycropml.algorithm)}

\begin{fulllineitems}
\phantomsection\label{\detokenize{_dvlpt/pycropml:pycropml.algorithm.Algorithm}}\pysiglinewithargsret{\sphinxbfcode{class }\sphinxcode{pycropml.algorithm.}\sphinxbfcode{Algorithm}}{\emph{language}, \emph{development}}{}
Bases: \sphinxhref{https://docs.python.org/3.4/library/functions.html\#object}{\sphinxcode{object}}

\end{fulllineitems}



\subsubsection{pycropml.checking module}
\label{\detokenize{_dvlpt/pycropml:module-pycropml.checking}}\label{\detokenize{_dvlpt/pycropml:pycropml-checking-module}}\index{pycropml.checking (module)}\index{Test (class in pycropml.checking)}

\begin{fulllineitems}
\phantomsection\label{\detokenize{_dvlpt/pycropml:pycropml.checking.Test}}\pysiglinewithargsret{\sphinxbfcode{class }\sphinxcode{pycropml.checking.}\sphinxbfcode{Test}}{\emph{name}}{}
Bases: {\hyperref[\detokenize{_dvlpt/pycropml:pycropml.checking.Testset}]{\sphinxcrossref{\sphinxcode{pycropml.checking.Testset}}}}

\end{fulllineitems}

\index{Testset (class in pycropml.checking)}

\begin{fulllineitems}
\phantomsection\label{\detokenize{_dvlpt/pycropml:pycropml.checking.Testset}}\pysiglinewithargsret{\sphinxbfcode{class }\sphinxcode{pycropml.checking.}\sphinxbfcode{Testset}}{\emph{name}, \emph{parameterset}, \emph{description}, \emph{uri=None}}{}
Bases: \sphinxhref{https://docs.python.org/3.4/library/functions.html\#object}{\sphinxcode{object}}

Test

\end{fulllineitems}

\index{testset() (in module pycropml.checking)}

\begin{fulllineitems}
\phantomsection\label{\detokenize{_dvlpt/pycropml:pycropml.checking.testset}}\pysiglinewithargsret{\sphinxcode{pycropml.checking.}\sphinxbfcode{testset}}{\emph{model}, \emph{name}, \emph{kwds}}{}
\end{fulllineitems}



\subsubsection{pycropml.description module}
\label{\detokenize{_dvlpt/pycropml:pycropml-description-module}}\label{\detokenize{_dvlpt/pycropml:module-pycropml.description}}\index{pycropml.description (module)}\index{Description (class in pycropml.description)}

\begin{fulllineitems}
\phantomsection\label{\detokenize{_dvlpt/pycropml:pycropml.description.Description}}\pysigline{\sphinxbfcode{class }\sphinxcode{pycropml.description.}\sphinxbfcode{Description}}
Bases: \sphinxhref{https://docs.python.org/3.4/library/functions.html\#object}{\sphinxcode{object}}

Model Unit Description.
\begin{description}
\item[{A description is defined by:}] \leavevmode\begin{itemize}
\item {} 
Title

\item {} 
Author

\item {} 
Institution

\item {} 
Reference

\item {} 
Abstract

\end{itemize}

\end{description}

\end{fulllineitems}



\subsubsection{pycropml.inout module}
\label{\detokenize{_dvlpt/pycropml:pycropml-inout-module}}\label{\detokenize{_dvlpt/pycropml:module-pycropml.inout}}\index{pycropml.inout (module)}\index{Input (class in pycropml.inout)}

\begin{fulllineitems}
\phantomsection\label{\detokenize{_dvlpt/pycropml:pycropml.inout.Input}}\pysiglinewithargsret{\sphinxbfcode{class }\sphinxcode{pycropml.inout.}\sphinxbfcode{Input}}{\emph{kwds}}{}
Bases: {\hyperref[\detokenize{_dvlpt/pycropml:pycropml.inout.InputOutput}]{\sphinxcrossref{\sphinxcode{pycropml.inout.InputOutput}}}}

\end{fulllineitems}

\index{InputOutput (class in pycropml.inout)}

\begin{fulllineitems}
\phantomsection\label{\detokenize{_dvlpt/pycropml:pycropml.inout.InputOutput}}\pysiglinewithargsret{\sphinxbfcode{class }\sphinxcode{pycropml.inout.}\sphinxbfcode{InputOutput}}{\emph{kwds}}{}
Bases: \sphinxhref{https://docs.python.org/3.4/library/functions.html\#object}{\sphinxcode{object}}

\end{fulllineitems}

\index{Output (class in pycropml.inout)}

\begin{fulllineitems}
\phantomsection\label{\detokenize{_dvlpt/pycropml:pycropml.inout.Output}}\pysiglinewithargsret{\sphinxbfcode{class }\sphinxcode{pycropml.inout.}\sphinxbfcode{Output}}{\emph{kwds}}{}
Bases: {\hyperref[\detokenize{_dvlpt/pycropml:pycropml.inout.InputOutput}]{\sphinxcrossref{\sphinxcode{pycropml.inout.InputOutput}}}}

\end{fulllineitems}



\subsubsection{pycropml.modelunit module}
\label{\detokenize{_dvlpt/pycropml:pycropml-modelunit-module}}\label{\detokenize{_dvlpt/pycropml:module-pycropml.modelunit}}\index{pycropml.modelunit (module)}\index{ModelDefinition (class in pycropml.modelunit)}

\begin{fulllineitems}
\phantomsection\label{\detokenize{_dvlpt/pycropml:pycropml.modelunit.ModelDefinition}}\pysiglinewithargsret{\sphinxbfcode{class }\sphinxcode{pycropml.modelunit.}\sphinxbfcode{ModelDefinition}}{\emph{kwds}}{}
Bases: \sphinxhref{https://docs.python.org/3.4/library/functions.html\#object}{\sphinxcode{object}}

\end{fulllineitems}

\index{ModelUnit (class in pycropml.modelunit)}

\begin{fulllineitems}
\phantomsection\label{\detokenize{_dvlpt/pycropml:pycropml.modelunit.ModelUnit}}\pysiglinewithargsret{\sphinxbfcode{class }\sphinxcode{pycropml.modelunit.}\sphinxbfcode{ModelUnit}}{\emph{kwds}}{}
Bases: \sphinxcode{pycropml.modelunit.ModelDefinition}

Formal description of a Model Unit.
\index{add\_description() (pycropml.modelunit.ModelUnit method)}

\begin{fulllineitems}
\phantomsection\label{\detokenize{_dvlpt/pycropml:pycropml.modelunit.ModelUnit.add_description}}\pysiglinewithargsret{\sphinxbfcode{add\_description}}{\emph{description}}{}
TODO

\end{fulllineitems}


\end{fulllineitems}



\subsubsection{pycropml.parameterset module}
\label{\detokenize{_dvlpt/pycropml:module-pycropml.parameterset}}\label{\detokenize{_dvlpt/pycropml:pycropml-parameterset-module}}\index{pycropml.parameterset (module)}\index{Parameterset (class in pycropml.parameterset)}

\begin{fulllineitems}
\phantomsection\label{\detokenize{_dvlpt/pycropml:pycropml.parameterset.Parameterset}}\pysiglinewithargsret{\sphinxbfcode{class }\sphinxcode{pycropml.parameterset.}\sphinxbfcode{Parameterset}}{\emph{name}, \emph{description}, \emph{uri=None}}{}
Bases: \sphinxhref{https://docs.python.org/3.4/library/functions.html\#object}{\sphinxcode{object}}

Parameter set

\end{fulllineitems}

\index{parameterset() (in module pycropml.parameterset)}

\begin{fulllineitems}
\phantomsection\label{\detokenize{_dvlpt/pycropml:pycropml.parameterset.parameterset}}\pysiglinewithargsret{\sphinxcode{pycropml.parameterset.}\sphinxbfcode{parameterset}}{\emph{model}, \emph{name}, \emph{kwds}}{}
\end{fulllineitems}



\subsubsection{pycropml.pparse module}
\label{\detokenize{_dvlpt/pycropml:module-pycropml.pparse}}\label{\detokenize{_dvlpt/pycropml:pycropml-pparse-module}}\index{pycropml.pparse (module)}
License, Header
\index{ModelParser (class in pycropml.pparse)}

\begin{fulllineitems}
\phantomsection\label{\detokenize{_dvlpt/pycropml:pycropml.pparse.ModelParser}}\pysigline{\sphinxbfcode{class }\sphinxcode{pycropml.pparse.}\sphinxbfcode{ModelParser}}
Bases: {\hyperref[\detokenize{_dvlpt/pycropml:pycropml.pparse.Parser}]{\sphinxcrossref{\sphinxcode{pycropml.pparse.Parser}}}}

Read an XML file and transform it in our object model.
\index{Algorithm() (pycropml.pparse.ModelParser method)}

\begin{fulllineitems}
\phantomsection\label{\detokenize{_dvlpt/pycropml:pycropml.pparse.ModelParser.Algorithm}}\pysiglinewithargsret{\sphinxbfcode{Algorithm}}{\emph{elt}}{}
\end{fulllineitems}

\index{Description() (pycropml.pparse.ModelParser method)}

\begin{fulllineitems}
\phantomsection\label{\detokenize{_dvlpt/pycropml:pycropml.pparse.ModelParser.Description}}\pysiglinewithargsret{\sphinxbfcode{Description}}{\emph{Title}, \emph{Author}, \emph{Institution}, \emph{Reference}, \emph{Abstract}}{}
\end{fulllineitems}

\index{Input() (pycropml.pparse.ModelParser method)}

\begin{fulllineitems}
\phantomsection\label{\detokenize{_dvlpt/pycropml:pycropml.pparse.ModelParser.Input}}\pysiglinewithargsret{\sphinxbfcode{Input}}{\emph{elts}}{}
\end{fulllineitems}

\index{Inputs() (pycropml.pparse.ModelParser method)}

\begin{fulllineitems}
\phantomsection\label{\detokenize{_dvlpt/pycropml:pycropml.pparse.ModelParser.Inputs}}\pysiglinewithargsret{\sphinxbfcode{Inputs}}{\emph{Input}}{}
\end{fulllineitems}

\index{ModelUnit() (pycropml.pparse.ModelParser method)}

\begin{fulllineitems}
\phantomsection\label{\detokenize{_dvlpt/pycropml:pycropml.pparse.ModelParser.ModelUnit}}\pysiglinewithargsret{\sphinxbfcode{ModelUnit}}{\emph{elts}}{}
ModelUnit (Description,Inputs,Outputs,Algorithm,Parametersets,
Testsets)

\end{fulllineitems}

\index{Output() (pycropml.pparse.ModelParser method)}

\begin{fulllineitems}
\phantomsection\label{\detokenize{_dvlpt/pycropml:pycropml.pparse.ModelParser.Output}}\pysiglinewithargsret{\sphinxbfcode{Output}}{\emph{elts}}{}
\end{fulllineitems}

\index{Outputs() (pycropml.pparse.ModelParser method)}

\begin{fulllineitems}
\phantomsection\label{\detokenize{_dvlpt/pycropml:pycropml.pparse.ModelParser.Outputs}}\pysiglinewithargsret{\sphinxbfcode{Outputs}}{\emph{elts}}{}
Ouputs (Output)

\end{fulllineitems}

\index{Parameterset() (pycropml.pparse.ModelParser method)}

\begin{fulllineitems}
\phantomsection\label{\detokenize{_dvlpt/pycropml:pycropml.pparse.ModelParser.Parameterset}}\pysiglinewithargsret{\sphinxbfcode{Parameterset}}{\emph{elts}}{}
\end{fulllineitems}

\index{Parametersets() (pycropml.pparse.ModelParser method)}

\begin{fulllineitems}
\phantomsection\label{\detokenize{_dvlpt/pycropml:pycropml.pparse.ModelParser.Parametersets}}\pysiglinewithargsret{\sphinxbfcode{Parametersets}}{\emph{Parameterset}}{}
\end{fulllineitems}

\index{Testset() (pycropml.pparse.ModelParser method)}

\begin{fulllineitems}
\phantomsection\label{\detokenize{_dvlpt/pycropml:pycropml.pparse.ModelParser.Testset}}\pysiglinewithargsret{\sphinxbfcode{Testset}}{\emph{Test}}{}
\end{fulllineitems}

\index{Testsets() (pycropml.pparse.ModelParser method)}

\begin{fulllineitems}
\phantomsection\label{\detokenize{_dvlpt/pycropml:pycropml.pparse.ModelParser.Testsets}}\pysiglinewithargsret{\sphinxbfcode{Testsets}}{\emph{Testset}}{}
\end{fulllineitems}

\index{dispatch() (pycropml.pparse.ModelParser method)}

\begin{fulllineitems}
\phantomsection\label{\detokenize{_dvlpt/pycropml:pycropml.pparse.ModelParser.dispatch}}\pysiglinewithargsret{\sphinxbfcode{dispatch}}{\emph{elt}}{}
\end{fulllineitems}

\index{param() (pycropml.pparse.ModelParser method)}

\begin{fulllineitems}
\phantomsection\label{\detokenize{_dvlpt/pycropml:pycropml.pparse.ModelParser.param}}\pysiglinewithargsret{\sphinxbfcode{param}}{\emph{pset}, \emph{elt}}{}
Param

\end{fulllineitems}

\index{parse() (pycropml.pparse.ModelParser method)}

\begin{fulllineitems}
\phantomsection\label{\detokenize{_dvlpt/pycropml:pycropml.pparse.ModelParser.parse}}\pysiglinewithargsret{\sphinxbfcode{parse}}{\emph{fn}}{}
\end{fulllineitems}


\end{fulllineitems}

\index{Parser (class in pycropml.pparse)}

\begin{fulllineitems}
\phantomsection\label{\detokenize{_dvlpt/pycropml:pycropml.pparse.Parser}}\pysigline{\sphinxbfcode{class }\sphinxcode{pycropml.pparse.}\sphinxbfcode{Parser}}
Bases: \sphinxhref{https://docs.python.org/3.4/library/functions.html\#object}{\sphinxcode{object}}

Read an XML file and transform it in our object model.
\index{dispatch() (pycropml.pparse.Parser method)}

\begin{fulllineitems}
\phantomsection\label{\detokenize{_dvlpt/pycropml:pycropml.pparse.Parser.dispatch}}\pysiglinewithargsret{\sphinxbfcode{dispatch}}{\emph{elt}}{}
\end{fulllineitems}

\index{parse() (pycropml.pparse.Parser method)}

\begin{fulllineitems}
\phantomsection\label{\detokenize{_dvlpt/pycropml:pycropml.pparse.Parser.parse}}\pysiglinewithargsret{\sphinxbfcode{parse}}{\emph{fn}}{}
\end{fulllineitems}


\end{fulllineitems}

\index{model\_parser() (in module pycropml.pparse)}

\begin{fulllineitems}
\phantomsection\label{\detokenize{_dvlpt/pycropml:pycropml.pparse.model_parser}}\pysiglinewithargsret{\sphinxcode{pycropml.pparse.}\sphinxbfcode{model\_parser}}{\emph{fn}}{}
Parse a set of models as xml files and return the models.

Returns ModelUnit object of the CropML Model.

\end{fulllineitems}



\subsubsection{pycropml.render\_notebook module}
\label{\detokenize{_dvlpt/pycropml:module-pycropml.render_notebook}}\label{\detokenize{_dvlpt/pycropml:pycropml-render-notebook-module}}\index{pycropml.render\_notebook (module)}
License, Header

Use pkglts

Problems:
- name of a model unit?
\index{Model2Nb (class in pycropml.render\_notebook)}

\begin{fulllineitems}
\phantomsection\label{\detokenize{_dvlpt/pycropml:pycropml.render_notebook.Model2Nb}}\pysiglinewithargsret{\sphinxbfcode{class }\sphinxcode{pycropml.render\_notebook.}\sphinxbfcode{Model2Nb}}{\emph{models}, \emph{dir=None}}{}
Bases: {\hyperref[\detokenize{_dvlpt/pycropml:pycropml.render_python.Model2Package}]{\sphinxcrossref{\sphinxcode{pycropml.render\_python.Model2Package}}}}

Generate a Jupyter Notebook from a set of models.
\index{generate\_notebook() (pycropml.render\_notebook.Model2Nb method)}

\begin{fulllineitems}
\phantomsection\label{\detokenize{_dvlpt/pycropml:pycropml.render_notebook.Model2Nb.generate_notebook}}\pysiglinewithargsret{\sphinxbfcode{generate\_notebook}}{}{}
Generate a Python package equivalent to the xml definition.

Args:
- models : a list of model
- dir: the directory where the code is generated.

Returns:
- None or status

\end{fulllineitems}

\index{generate\_test() (pycropml.render\_notebook.Model2Nb method)}

\begin{fulllineitems}
\phantomsection\label{\detokenize{_dvlpt/pycropml:pycropml.render_notebook.Model2Nb.generate_test}}\pysiglinewithargsret{\sphinxbfcode{generate\_test}}{\emph{model\_unit}}{}
\end{fulllineitems}

\index{run() (pycropml.render\_notebook.Model2Nb method)}

\begin{fulllineitems}
\phantomsection\label{\detokenize{_dvlpt/pycropml:pycropml.render_notebook.Model2Nb.run}}\pysiglinewithargsret{\sphinxbfcode{run}}{}{}
TODO.

\end{fulllineitems}


\end{fulllineitems}



\subsubsection{pycropml.render\_python module}
\label{\detokenize{_dvlpt/pycropml:pycropml-render-python-module}}\label{\detokenize{_dvlpt/pycropml:module-pycropml.render_python}}\index{pycropml.render\_python (module)}
License, Header

Use pkglts

Problems:
- name of a model unit?
\index{Model2Package (class in pycropml.render\_python)}

\begin{fulllineitems}
\phantomsection\label{\detokenize{_dvlpt/pycropml:pycropml.render_python.Model2Package}}\pysiglinewithargsret{\sphinxbfcode{class }\sphinxcode{pycropml.render\_python.}\sphinxbfcode{Model2Package}}{\emph{models}, \emph{dir=None}}{}
Bases: \sphinxhref{https://docs.python.org/3.4/library/functions.html\#object}{\sphinxcode{object}}

TODO
\index{DATATYPE (pycropml.render\_python.Model2Package attribute)}

\begin{fulllineitems}
\phantomsection\label{\detokenize{_dvlpt/pycropml:pycropml.render_python.Model2Package.DATATYPE}}\pysigline{\sphinxbfcode{DATATYPE}\sphinxbfcode{ = \{'DOUBLEARRAY': \textless{}built-in function array\textgreater{}, 'Double': \textless{}type 'float'\textgreater{}, 'int': \textless{}type 'int'\textgreater{}, 'real': \textless{}type 'float'\textgreater{}, 'string': \textless{}type 'str'\textgreater{}\}}}
\end{fulllineitems}

\index{generate\_algorithm() (pycropml.render\_python.Model2Package method)}

\begin{fulllineitems}
\phantomsection\label{\detokenize{_dvlpt/pycropml:pycropml.render_python.Model2Package.generate_algorithm}}\pysiglinewithargsret{\sphinxbfcode{generate\_algorithm}}{\emph{model\_unit}}{}
\end{fulllineitems}

\index{generate\_component() (pycropml.render\_python.Model2Package method)}

\begin{fulllineitems}
\phantomsection\label{\detokenize{_dvlpt/pycropml:pycropml.render_python.Model2Package.generate_component}}\pysiglinewithargsret{\sphinxbfcode{generate\_component}}{\emph{model\_unit}}{}
Todo

\end{fulllineitems}

\index{generate\_function\_doc() (pycropml.render\_python.Model2Package method)}

\begin{fulllineitems}
\phantomsection\label{\detokenize{_dvlpt/pycropml:pycropml.render_python.Model2Package.generate_function_doc}}\pysiglinewithargsret{\sphinxbfcode{generate\_function\_doc}}{\emph{model\_unit}}{}
\end{fulllineitems}

\index{generate\_function\_signature() (pycropml.render\_python.Model2Package method)}

\begin{fulllineitems}
\phantomsection\label{\detokenize{_dvlpt/pycropml:pycropml.render_python.Model2Package.generate_function_signature}}\pysiglinewithargsret{\sphinxbfcode{generate\_function\_signature}}{\emph{model\_unit}}{}
\end{fulllineitems}

\index{generate\_package() (pycropml.render\_python.Model2Package method)}

\begin{fulllineitems}
\phantomsection\label{\detokenize{_dvlpt/pycropml:pycropml.render_python.Model2Package.generate_package}}\pysiglinewithargsret{\sphinxbfcode{generate\_package}}{}{}
Generate a Python package equivalent to the xml definition.

Args:
- models : a list of model
- dir: the directory where the code is generated.

Returns:
- None or status

\end{fulllineitems}

\index{generate\_test() (pycropml.render\_python.Model2Package method)}

\begin{fulllineitems}
\phantomsection\label{\detokenize{_dvlpt/pycropml:pycropml.render_python.Model2Package.generate_test}}\pysiglinewithargsret{\sphinxbfcode{generate\_test}}{\emph{model\_unit}}{}
\end{fulllineitems}

\index{num (pycropml.render\_python.Model2Package attribute)}

\begin{fulllineitems}
\phantomsection\label{\detokenize{_dvlpt/pycropml:pycropml.render_python.Model2Package.num}}\pysigline{\sphinxbfcode{num}\sphinxbfcode{ = 0}}
\end{fulllineitems}

\index{run() (pycropml.render\_python.Model2Package method)}

\begin{fulllineitems}
\phantomsection\label{\detokenize{_dvlpt/pycropml:pycropml.render_python.Model2Package.run}}\pysiglinewithargsret{\sphinxbfcode{run}}{}{}
TODO.

\end{fulllineitems}

\index{write\_tests() (pycropml.render\_python.Model2Package method)}

\begin{fulllineitems}
\phantomsection\label{\detokenize{_dvlpt/pycropml:pycropml.render_python.Model2Package.write_tests}}\pysiglinewithargsret{\sphinxbfcode{write\_tests}}{}{}
TODO: Manage several models rather than just one.

\end{fulllineitems}


\end{fulllineitems}



\subsubsection{pycropml.version module}
\label{\detokenize{_dvlpt/pycropml:pycropml-version-module}}\label{\detokenize{_dvlpt/pycropml:module-pycropml.version}}\index{pycropml.version (module)}
Maintain version for this package.
Do not edit this file, use ‘version’ section of config.
\index{MAJOR (in module pycropml.version)}

\begin{fulllineitems}
\phantomsection\label{\detokenize{_dvlpt/pycropml:pycropml.version.MAJOR}}\pysigline{\sphinxcode{pycropml.version.}\sphinxbfcode{MAJOR}\sphinxbfcode{ = 0}}
(int) Version major component.

\end{fulllineitems}

\index{MINOR (in module pycropml.version)}

\begin{fulllineitems}
\phantomsection\label{\detokenize{_dvlpt/pycropml:pycropml.version.MINOR}}\pysigline{\sphinxcode{pycropml.version.}\sphinxbfcode{MINOR}\sphinxbfcode{ = 0}}
(int) Version minor component.

\end{fulllineitems}

\index{POST (in module pycropml.version)}

\begin{fulllineitems}
\phantomsection\label{\detokenize{_dvlpt/pycropml:pycropml.version.POST}}\pysigline{\sphinxcode{pycropml.version.}\sphinxbfcode{POST}\sphinxbfcode{ = 2}}
(int) Version post or bugfix component.

\end{fulllineitems}



\subsubsection{Module contents}
\label{\detokenize{_dvlpt/pycropml:module-contents}}\label{\detokenize{_dvlpt/pycropml:module-pycropml}}\index{pycropml (module)}

\section{\sphinxstylestrong{Usecases}}
\label{\detokenize{user/usecases:usecases}}\label{\detokenize{user/usecases::doc}}

\section{\sphinxstylestrong{Licence}}
\label{\detokenize{user/license::doc}}\label{\detokenize{user/license:licence}}
PyCropML is released under a MIT License.


\section{\sphinxstylestrong{Usecases}}
\label{\detokenize{user/Publication:usecases}}\label{\detokenize{user/Publication::doc}}

\section{\sphinxstylestrong{Glossary}}
\label{\detokenize{user/glossary:glossary}}\label{\detokenize{user/glossary::doc}}
Terminology
\begin{description}
\item[{Model\index{Model|textbf}}] \leavevmode\phantomsection\label{\detokenize{user/glossary:term-model}}
Simplified representation of the crop system within specific objectives.

\end{description}


\chapter{Credits}
\label{\detokenize{index:credits}}
\sphinxstylestrong{1. Development Lead}
\begin{itemize}
\item {} 
Cyrille Ahmed Midingoyi, \textless{}\sphinxhref{mailto:cyrille.midingoyi@inra.fr}{cyrille.midingoyi@inra.fr}\textgreater{}

\item {} 
Christophe Pradal, \textless{}\sphinxhref{mailto:christophe.pradal@cirad.fr}{christophe.pradal@cirad.fr}\textgreater{}

\end{itemize}

\sphinxstylestrong{2. Contributors}

None yet. Why not be the first?


\section{History}
\label{\detokenize{index:history}}

\subsection{creation (2018-01-18)}
\label{\detokenize{index:creation-2018-01-18}}\begin{itemize}
\item {} 
First release on PyPI.

\end{itemize}


\section{\sphinxstylestrong{Indices and tables}}
\label{\detokenize{index:indices-and-tables}}\begin{itemize}
\item {} 
\DUrole{xref,std,std-ref}{genindex}

\item {} 
\DUrole{xref,std,std-ref}{modindex}

\item {} 
\DUrole{xref,std,std-ref}{search}

\end{itemize}


\section{\sphinxstylestrong{Supported by:}}
\label{\detokenize{index:supported-by}}
\begin{figure}[htbp]
\centering
\sphinxhref{http://openalea.gforge.inria.fr/dokuwiki/doku.php}{\sphinxincludegraphics[width=0.150\linewidth]{{openalea}.png}}\end{figure}

\begin{figure}[htbp]
\centering
\sphinxhref{https://www6.inra.fr/record}{\sphinxincludegraphics[width=0.150\linewidth]{{record}.jpg}}\end{figure}

\begin{figure}[htbp]
\centering
\sphinxhref{http://bioma.jrc.ec.europa.eu/components/componentstools/bioma/WebHelp/index.htm}{\sphinxincludegraphics[width=0.150\linewidth]{{bioma}.png}}\end{figure}

\begin{figure}[htbp]
\centering
\sphinxhref{http://www1.clermont.inra.fr/siriusquality/}{\sphinxincludegraphics[width=0.150\linewidth]{{siriusquality}.png}}\end{figure}

\begin{figure}[htbp]
\centering
\sphinxhref{http://www.simplace.net/Joomla/}{\sphinxincludegraphics[width=0.150\linewidth]{{simplace}.png}}\end{figure}


\renewcommand{\indexname}{Python Module Index}
\begin{sphinxtheindex}
\def\bigletter#1{{\Large\sffamily#1}\nopagebreak\vspace{1mm}}
\bigletter{p}
\item {\sphinxstyleindexentry{pycropml}}\sphinxstyleindexpageref{_dvlpt/pycropml:\detokenize{module-pycropml}}
\item {\sphinxstyleindexentry{pycropml.algorithm}}\sphinxstyleindexpageref{_dvlpt/pycropml:\detokenize{module-pycropml.algorithm}}
\item {\sphinxstyleindexentry{pycropml.checking}}\sphinxstyleindexpageref{_dvlpt/pycropml:\detokenize{module-pycropml.checking}}
\item {\sphinxstyleindexentry{pycropml.description}}\sphinxstyleindexpageref{_dvlpt/pycropml:\detokenize{module-pycropml.description}}
\item {\sphinxstyleindexentry{pycropml.inout}}\sphinxstyleindexpageref{_dvlpt/pycropml:\detokenize{module-pycropml.inout}}
\item {\sphinxstyleindexentry{pycropml.parameterset}}\sphinxstyleindexpageref{_dvlpt/pycropml:\detokenize{module-pycropml.parameterset}}
\item {\sphinxstyleindexentry{pycropml.pparse}}\sphinxstyleindexpageref{_dvlpt/pycropml:\detokenize{module-pycropml.pparse}}
\item {\sphinxstyleindexentry{pycropml.render\_notebook}}\sphinxstyleindexpageref{_dvlpt/pycropml:\detokenize{module-pycropml.render_notebook}}
\item {\sphinxstyleindexentry{pycropml.render\_python}}\sphinxstyleindexpageref{_dvlpt/pycropml:\detokenize{module-pycropml.render_python}}
\item {\sphinxstyleindexentry{pycropml.version}}\sphinxstyleindexpageref{_dvlpt/pycropml:\detokenize{module-pycropml.version}}
\end{sphinxtheindex}

\renewcommand{\indexname}{Index}
\printindex
\end{document}